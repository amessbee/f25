\documentclass[12pt]{article}
\usepackage[margin=1in]{geometry}
\usepackage{titlesec}
\usepackage{enumitem}
\usepackage{hyperref}

\title{Course Outline: Social Network Analysis I (SNA-I)}
\author{Instructor: [Dr. Mudassir Shabbir]}
\date{Fall Semester, [2025]}

\begin{document}

\maketitle

\section*{Course Overview}
\begin{itemize}[leftmargin=1.5em]
  \item \textbf{Level:} 300-level undergraduate
  \item \textbf{Credits:} 3
  \item \textbf{Prerequisite:} Algorithms course with grade B- or higher
  \item \textbf{Eligibility:} Open to undergraduate students. Graduate students may attend informally but cannot take the course for credit.
  \item \textbf{Schedule:} Lectures will be scheduled within the 11:00 AM -- 2:00 PM window.
  \item \textbf{Format:} Hands-on lectures, labs, homeworks (front-loaded), group activities, and final project
\end{itemize}

\section*{Learning Objectives}
By the end of the course, students will:
\begin{itemize}
  \item Understand and apply core concepts in supervised and unsupervised machine learning
  \item Gain proficiency in Python for machine learning and data analysis
  \item Model, analyze, and visualize real-world networks using graph theory
  \item Apply machine learning techniques to network problems such as link prediction and node classification
  \item Collaborate on applied projects and communicate results effectively
\end{itemize}

\section*{Weekly Breakdown}

\subsection*{Unit I: Machine Learning Foundations (Weeks 1--6)}

\subsubsection*{Week 1: Python Fundamentals \& Data Handling}
\begin{itemize}
  \item Python review, Jupyter/Colab setup
  \item Using \texttt{pandas} for data cleaning and summaries
  \item \textbf{Lab:} Clean and summarize a real-world dataset
  \item \textbf{Homework 1:} Explore and clean a dataset (UCI/Kaggle); compute stats and visualize
\end{itemize}

\subsubsection*{Week 2: Supervised Learning -- Classification}
\begin{itemize}
  \item Train/test splits, decision trees, k-NN
  \item Accuracy, precision, recall, F1 score
  \item \textbf{Lab:} Build classifiers on Titanic or similar
  \item \textbf{Homework 2:} Compare classifiers on a binary classification dataset, include visualizations
\end{itemize}

\subsubsection*{Week 3: Regression \& Feature Engineering}
\begin{itemize}
  \item Linear regression, polynomial regression
  \item Feature scaling, encoding, binning
  \item \textbf{Lab:} Predict housing prices with engineered features
  \item \textbf{Homework 3:} Train a regressor with at least 3 engineered features; evaluate performance
\end{itemize}

\subsubsection*{Week 4: Model Evaluation \& Pipelines}
\begin{itemize}
  \item Cross-validation, bias-variance tradeoff
  \item \texttt{sklearn} pipelines, reusable components
  \item \textbf{Lab:} Build and test classification pipelines
  \item \textbf{Homework 4:} Design a full pipeline with evaluation and visual reporting
\end{itemize}

\subsubsection*{Week 5: Clustering \& Unsupervised Learning}
\begin{itemize}
  \item k-means, hierarchical clustering, silhouette score
  \item \textbf{Lab:} Cluster student or consumer datasets
  \item \textbf{Group Activity:} Present clustering results and interpretations
\end{itemize}

\subsubsection*{Week 6: Dimensionality Reduction}
\begin{itemize}
  \item PCA, t-SNE, embeddings intro
  \item \textbf{Lab:} Visualize projections in 2D
  \item \textbf{Optional Homework 5:} Apply dimensionality reduction and cluster/visualize results
\end{itemize}

\subsection*{Unit II: Network Analysis and Graph ML (Weeks 7--14)}

\subsubsection*{Week 7: Introduction to Graphs}
\begin{itemize}
  \item Graph terminology, adjacency matrix/list, edge lists
  \item \textbf{Lab:} Load and visualize networks in \texttt{networkx}
\end{itemize}

\subsubsection*{Week 8: Centrality Measures}
\begin{itemize}
  \item Degree, betweenness, closeness, eigenvector
  \item \textbf{Lab:} Analyze real-world graphs for key nodes
\end{itemize}

\subsubsection*{Week 9: Communities \& Subgraphs}
\begin{itemize}
  \item Louvain, label propagation, modularity
  \item \textbf{Lab:} Community detection in ego networks
\end{itemize}

\subsubsection*{Week 10: Network Diffusion}
\begin{itemize}
  \item SI/SIR models, cascade simulations
  \item \textbf{Lab:} Simulate influence spread in social graphs
\end{itemize}

\subsubsection*{Week 11: Link Prediction \& Node Classification}
\begin{itemize}
  \item Heuristics (Jaccard, Adamic-Adar), ML classifiers
  \item \textbf{Lab:} Predict edges using logistic regression
\end{itemize}

\subsubsection*{Week 12: Dynamic Networks}
\begin{itemize}
  \item Temporal graphs, snapshots, growth modeling
  \item \textbf{Lab:} Analyze change over time in collaboration/email networks
\end{itemize}

\subsubsection*{Weeks 13--14: Final Projects}
\begin{itemize}
  \item Student presentations and demos
  \item \textbf{Deliverables:} Code notebook, slides, 2-page summary
\end{itemize}

\section*{Assessment Breakdown}

\begin{tabular}{|l|c|}
\hline
\textbf{Component} & \textbf{Weight} \\
\hline
Homework Assignments (Weeks 1--5) & 35\% \\
Labs and Participation & 15\% \\
Group Activities (Weeks 5, 9) & 10\% \\
Final Project & 30\% \\
Peer Review/Reflection & 10\% \\
\hline
\end{tabular}

\section*{Tools and Platforms}
\begin{itemize}
  \item Python: \texttt{pandas}, \texttt{matplotlib}, \texttt{seaborn}, \texttt{scikit-learn}, \texttt{networkx}
  \item Platforms: Google Colab / Jupyter Notebooks
  \item Optional: Gephi, PyTorch Geometric, HuggingFace Datasets
\end{itemize}

\section*{Example Project Ideas}
\begin{itemize}
  \item Modeling diffusion of memes in Twitter networks
  \item Link prediction in a GitHub or DBLP collaboration graph
  \item Detecting clusters in Discord server graphs
  \item Visualizing temporal evolution of a citation network
  \item Combining PCA and community detection for student grouping
\end{itemize}

\end{document}
